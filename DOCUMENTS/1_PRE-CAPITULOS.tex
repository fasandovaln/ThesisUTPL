% ====================================================================
%%- Información PRE-CAPÍTULOS -%%
% ====================================================================
% Creado por: Francisco Alberto Sandoval Noreña
% e-mail: fasandoval@utpl.edu.ec
% Fecha: 11 de noviembre de 2018
% Version: 0.1
% ====================================================================
%
%%- Página: Carátula -%% ---------------------------------------------
\paginaCaratula

%%- Página: Aprobación del Director del Trabajo de Titulación -%% ----
\paginaAprobacion

%%- Página: Declaración de Autoría y Cesión de Derechos -%% ----------
\paginaAutoria

%%- Página: Dedicatoria -%% ------------------------------------------
\paginaDedicatoria{%
	\textcolor{blue}{\textit{En este espacio puedes ubicar la dedicatoria. Se ha agregado texto de relleno a continuación, para verificación del formato.}} \\
	\blindtext
}%							

%%- Página: Agradecimiento -%% ---------------------------------------
\paginaAgradecimiento{%
	\textcolor{blue}{\textit{En este espacio puedes ubicar los agradecimientos. Se ha agregado texto de relleno a continuación, para verificación del formato.}} \\
	\blindtext
}%

%%- Página: Lista de Contenidos, Lista de Figuras, Lista de Tablas -%%
\paginaListas

%%- Página: Resumen -%% ----------------------------------------------
\paginaResumen{%
	% Escribir resumen
	\textcolor{blue}{\textit{El resumen se presentará en un máximo 180 palabras, debe ser ubicado en la ficha del Senescyt; sintetiza el aporte que brinda el trabajo realizado.
	Obligatoriamente deberá contener las palabras claves (máximo tres).
	Se ha agregado texto de relleno a continuación, para verificación del formato.}} \\
	\blindtext
}%	
	% Ingreso de las palabras claves								
{Palabras, claves, Trabajo de Titulación}			
%
%%- Página: Abstract -%% ---------------------------------------------
\paginaAbstract{
	% Escribir abstract
	\textcolor{blue}{Abstract es el resumen traducido al idioma inglés en donde se incluyen las palabras claves.
	Obligatoriamente deberá contener las palabras claves (máximo tres).} \\
	\blindtext
}%								
	% Ingreso de Keywords
{Keywords, thesis, engineering}								

%%. Lista de Abreviaturas (opcional) -%% -----------------------------
	% Si no requiere usar ``Lista de Abreviaturas'' comente los comandos. 
\paginaListaAbrSim{Índice de abreviaturas}{2.5}%
{%
	\item [UTPL] 	Universidad Técnica Particular de Loja 
	\item [DCCE]	Departamento de Ciencias de la Computación y Electrónica 
	\item [SET]		Sección de Electrónica y Telecomunicaciones
	\item [MIMO]	\textit{Multiple-input multiple-output}
	\item [OFDM]	\textit{Orthogonal frequency division multiplexing}	
}
%
%%- Lista de Símbolos (Opcional) -%% ---------------------------------
	% Si no requiere usar ``Lista de Abreviaturas'' comente los comandos. 
\paginaListaAbrSim{Índice de símbolos}{2}%
{%
	\item [$ c $]			Speed of light						
	\item [$ \lambda $]		wavelength
	\item [$ \pi $]			number pi
	\item [\mbox{$E[\cdot]$}]	Expected value
}